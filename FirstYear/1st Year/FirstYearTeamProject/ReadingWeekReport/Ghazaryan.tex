\documentclass[a4paper]{report}
\usepackage[utf8]{inputenc}
\usepackage{url}
\renewcommand\thesection{\arabic{section}}
\title{Reading Week Report}
\author{Pavel Ghazaryan}
\date{November 2021}


\begin{document}
\begin{titlepage}
    \begin{center}

            \vspace*{1cm}
            \Huge
            \textsc{Reading Week Report}
            
            \vspace{0.5cm}
            \Large
            
            \vspace{1.5cm}
        
            \textbf{Pavel Ghazaryan(Group R3)}
            
            \vspace{1.5cm}
            \it{University of Manchester}
        
            \vfill
                    
                
             
    \end{center}
\end{titlepage}
\newpage
\newpage
\Large
\textbf{Questions}
\begin{enumerate}
    \item How your framework applies to the decision: does it give clear guidance on how to act in a similar situation? 
    \item How another framework applies to the decision?
\end{enumerate}
\newpage
\section{Decision 1: Collecting tweets for research paper without authors’ consent\cite{casestudy}}
\textbf{Ten Commandments of Computer Ethics\cite{tencommands}}\\\\
My Ethical framework is comparably vague for application in today’s developed world because it was conducted back in 1992 when there were no social media platforms and privacy/security issues. However, Commandment 7\cite{tencommands} can be applied to the above decision of using Tweets in a research paper without the permission of their authors. It states that users can use no computer resources without authorization. Tweets can be regarded as computer resources of Twitter accounts\cite{casestudy} in this case, meaning Students breached the Commandment 7 of Computer Ethics. \\\\
\textbf{ACM Code of Ethics\cite{ACM}}\\\\
The above-mentioned decision can be monitored again with ACM ethical framework but compared to the Ten Commandments, this time more specifically. Section 1.6 of the ACM framework talks about Respecting Privacy, which states, “Computing professionals should only use personal information for legitimate ends and without violating the rights of individuals and groups”\cite{ACM}. This means that when students collected the tweets they violated the right of individual users of Twitter without asking for permission.\\\\
\section{Decision 2: Students did not check with Feng "he [Feng] really wished they had consulted with him"\cite{casestudy}}
\textbf{Ten Commandments of Computer Ethics}\\\\
As mentioned earlier, due to its early formation, the Ten Commandments does not cover all of the nowadays-ethical problems. An example is the above-mentioned decision of not communicating with Feng which leads to the failure of the research paper. This kind of situation of unprofessional communication is not monitored at all in ten commandments as none of them includes information or guidance about professional communication meaning it’s neither ethical nor unethical from this point of view.\\\\
\textbf{ACS Code of Ethics\cite{ACS} }\\\\
On the other hand, the ACS framework has specific sections for communication to be competent and stay on the professional level. Section 4.10.4, suggests that students must cooperate with other professionals in order to advance in information processing\cite{ACS}. This means that the above-mentioned decision is unethical in terms of the framework because students did not communicate with Feng who is a professional in the research area.\\\\
\section{Decision 3: Bad/False data usage because of Test Corpus\cite{casestudy}}
\textbf{Ten Commandments of Computer Ethics}\\\\
Despite its early development, Ten Commandments have guidance for this decision. Specifically, Commandment 5\cite{tencommands} states that users should not contribute to the spread of misinformation. Test corpus results were used in the research paper to produce a conclusion. Test results are not false but using them as accurate data is the same as using incorrect data for research. This means that students contributed to the spread of false data, leading to a breach.\\\\
\textbf{The System Administrators' Code of Ethics\cite{SYS}}\\\\
Unfortunately, the System Administrators’ Code of Ethics does not offer any guidance or rule which should be followed in such cases. This means that the decision cannot be regarded as ethical or unethical in this framework. This is quite surprising because this framework has various points about being professional and following the rules but none of them includes anything about falsified data.
\bibliographystyle{plain}
\bibliography{ethics}
\end{document}

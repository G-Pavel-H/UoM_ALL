\documentclass{article}
\usepackage[utf8]{inputenc}
\usepackage{url}
\renewcommand\thesection{\arabic{section}}
\title{Reading Week Report}
\author{Pavel Ghazaryan}
\date{November 2021}


\begin{document}
\begin{titlepage}
    \begin{center}

            \vspace*{1cm}
            \Huge
            \textsc{Final Report}
            
            \vspace{0.5cm}
            \Large
            
            \vspace{1.5cm}
        
            \textbf{Pavel Ghazaryan(Group R3)}
            
            \vspace{1.5cm}
            \it{The University of Manchester}
        
            \vfill
                    
                
             
    \end{center}
\end{titlepage}
\newpage
\newpage
\Large
\textbf{Reflections}\\\


1.	One of the coding challenges I have overcome during the development of our Web application came up during the making of the read more page for each post. On the read more page for each post, we wanted to show the full contents of the post and its comments below it. This meant that the size of the div element containing the post should automatically adjust itself based on its contents. However, this was not happening and as soon as I was removing the height property from the CSS, the div element was just getting a height of zero, ignoring all of its contents.

\setlength{\parindent}{20pt} After spending countless hours and even days looking for explanations online, I have finally found my solution. As I have been using position relative and absolute properties for specifying the positions of the elements inside the div, the height could not be variable and could only be explicitly fixed through the code. That is when I have learnt about the display-block property which let me stack the main post above the comments and have an automatically adjustable height. Through this process, I have not only learnt more about CSS and web development but also understood that I should never give up and always work on reaching my goals. \\\\

2.	Throughout making our Web Application, we had team meetings each week to discuss our progress and make sure that everyone is on the same page. In order to easier cooperate as a team and complete the project in time, we assigned tasks to each group member. We were supposed to complete each task before a certain deadline and usually, that would be our next meeting. For example, for the main page where we would have our posts, we decided what are the tasks that need to be carried out. For example, making the categories section, making the posting section, the trending section and etc. Afterwards, everyone voluntarily would pick a task that they think they will be able to complete or are just interested in completing it. 

\setlength{\parindent}{20pt}This method helped us to always stay alongside our planned schedule and finish everything by the deadline. However, the most important factor of such planning was that all of my group members were very responsible and on time. We all were able to trust one another and be sure that the assigned tasks will be completed on time. That is why we finished our Web application as planned on time and earlier than the deadline. \\\\

3.	The one thing that I regret the most is that I did not participate much in the completion of some back-end tasks such as writing code in PHP or manipulating the database. I was always quite unsure if I should take tasks involving back-end as my only experience from before was with front-end. However, when in the final parts of our project I helped one of my group members in the creation of the delete function which would delete the specific post from the database, I understood that I was not so bad at it and that I should have given it a shot during the early times of the development as at that point the whole back-end was completed. 

\setlength{\parindent}{20pt} This experience made me to understand that sometimes it is important to step out from our comfort zone and take a risk in order to learn something new. I am glad that at least in the end of our project, I took that risk. From then on, I am actually looking for the tasks that are purely new to me as that is the only way to learn and develop new skills.



\end{document}

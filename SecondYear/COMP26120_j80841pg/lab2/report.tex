\documentclass[a4]{article}
\usepackage{gnuplottex}
\usepackage{csvsimple}
\usepackage{subcaption}
\title{COMP26120 Lab 2 Report}
\author{Your Name Here}
\begin{document}
\maketitle

\section{Experiment 1 -- Sorting Performance}

\subsection{Theoretical Best Case}

\paragraph{Hypothesis} The behaviour for insertion sort on sorted input is linear. 

%% Write a brief paragraph here to justify this hypothesis with reference to the theoretical complexity.

\paragraph{Experimental Design} 

%% Your experimental design should include the following

%% What data you generated

%% What program you ran on what combination

%% How you measured the time taken.

%% How you determined $f(x)$

%% Any scripts (not counting those we supplied) used to automated generation of data or generate results should be included in your git lab and you should state here where these scripts can be found.


\paragraph{Results} 

%% Ideally you should include a graph of your results showing the best fit line for f.

%% You should state the values you have determined for $c_1$ and $c_2$.

%% You should include the raw data as a table either here or in an appendix.  Raw timing data should also be included in your gitlab (just don't include the dictionaries and queries you generated and used).


\subsection{Theoretical Worst Case}

\paragraph{Hypothesis} The behaviour for insertion sort on reverse sorted input is quadratic.  

%% Write a brief paragraph here to justify this hypothesis with reference to the theoretical complexity.

\paragraph{Experimental Design} 

%% Your experimental design should include the following

%% What data you generated

%% What program you ran on what combination

%% How you measured the time taken.

%% How you determined $f(x)$

%% Any scripts (not counting those we supplied) used to automated generation of data or generate results should be included in your git lab and you should state here where these scripts can be found.


\paragraph{Results} 

%% Ideally you should include a graph of your results showing the best fit line for f.

%% You should state clearly whether your hypothesis was confirmed or refuted.

%% You should state the values you have determined for $c_1$ and $c_2$.

%% You should include the raw data as a table either here or in an appendix.  Raw timing data should also be included in your gitlab (just don't include the dictionaries and queries you generated and used).



\subsection{Average Case}

\paragraph{Hypothesis} The behaviour for insertion sort on random input is somewhere between the performance on sorted and reverse sorted input.  

%% Write a brief paragraph here to justify this hypothesis with reference to the theoretical complexity.

\paragraph{Experimental Design} 

%% Your experimental design should include the following

%% What data you generated

%% What program you ran on what combination

%% How you measured the time taken.

%% How you determined the best fit $f(x)$.  In this case you will have to describe how you decided whether a linear or quadratic line was a better fit.

%% Any scripts (not counting those we supplied) used to automated generation of data or generate results should be included in your git lab and you should state here where these scripts can be found.


\paragraph{Results} 

%% Ideally you should include a graph of your results showing the best fit line for f.  Ideally this graph will compare with the performance for the best and worst cases.

%% You should state clearly whether your hypothesis was confirmed or refuted.

%% You should state the values you have determined for $c_1$ and $c_2$.

%% You should include the raw data as a table either here or in an appendix.  Raw timing data should also be included in your gitlab (just don't include the dictionaries and queries you generated and used).


\subsection{Discussion}

%% Write a short paragraph discussing your results more generally.  Did sorted data turn out to be best case?  Did reverse sorted data turn out to be worst case?  Was the performance for average case more like best or worst case?  If things didn't behave as you expected can you speculate on why?  What further experiment(s) might you run to help determine why?

\section{Experiment 2}

\paragraph{Hypothesis} %% You want a hypothesis here probably expressed in terms of k (the dictionary size) and n (the query size) that suggests when you think it will become worth sorting.  You probably want to express this as a range of values and you will use your experiment to find exactly where in that range the point occurs.

%% You should then back the hypothesis up with some fuller discussion on how you computed the range where the crossover point was likely to occur based on your knowledge of the complexity of searching and sorting.

\paragraph{Experimental Design} 

%% Your experimental design should include the following

%% What parameters you varied 

%% What data you generated

%% What program you ran on what combination

%% How you measured the time taken.

%% How you determined the best fit $f(x)$ for each case.  

%% How you determined where sorting became "worth it"

%% Any scripts (not counting those we supplied) used to automated generation of data or generate results should be included in your git lab and you should state here where these scripts can be found.

\paragraph{Results} 

%% Ideally you should include a graph of your results.  You may need several graphs for this experiment showing several different set ups.

%% You should state clearly whether your hypothesis was confirmed or refuted.

%% You should state the value(s) you have determined for when sorting becomes worth it.

%% You should include the raw data as a table either here or in an appendix.  Raw timing data should also be included in your gitlab (just don't include the dictionaries and queries you generated and used).

\subsection{Discussion} %% Optional section

%% Depending on your results you may need some additional reflection -- for instance to hypothesise why things didn't behave as expected.  Or reflect on why you got different points where sorting became worth it for different set ups.

\appendix
%% As appendices you may want to include tables of raw data

%% scripts used (not including those we supplied) to generate dictionaries, queries and results.

\end{document}